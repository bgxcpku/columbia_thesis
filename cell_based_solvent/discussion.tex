%Summarize what you did and results
We have developed an application of a classic computer science grid based hashing algorithm to the implicit solvent model of PLOP.
We demonstrate that this application does not affect accuracy of results compared with the previous implicit solvent model implementation in PLOP.
Though in a small number of cases side chain conformations are predicted in widely different conformations by the hash based method and the old implementation, the two methods are equally likely to predict the more native-like structure, so this variance can be attributed to noise.
We have also found in other experiments that the final predicted structure of a minimization is very sensitive to both small changes in pre-minimization coordinates, of a magnitude far less than bond distances, and minimization parameters.
It possible that these effects magnify small differences present early in the experiment, resulting in much larger differences between the final predicted structures.
Finally, we present data showing that the reduced computational cost of evaluating the solvent contribution using this hash based approach dramatically reduces the total time spent evaluating the energy model by a factor of 1.6 to 2.5.

% talk about hash method, 
Note that any such geometric hashing will introduce some overhead for maintaining the data structure.
As discussed by Bentley and Friedman \cite{bentley1979data}, the total storage necessary for the hash structure and the time necessary to sort atoms into cells are both linear in the number of atoms, and placing or updating a single atom in the structure is a constant time operation.
The improvement in retrieval using this structure dominates the cost of maintaining the structure, and the difference becomes more pronounced as system size grows.
Bentley and Friedman also present a through review of the performance characteristics of a number of other geometric hashing techniques, though they compare the algorithms in a data agnostic means.
Taking advantage of the characteristics of physical data, in this case atomic coordinates, has some effect of the relative advantages and disadvantages of specific hashing techniques. 
Specifically, the maximum number of atoms per cell is limited by physical constraints of atomic interactions.
% other hashing methods - octree.  why yours is better
An octree is a similar, though hierarchical, hash structure used in computer graphics for fast location based retrieval.
However, because the criteria for ``collision'' in this case is a fixed distance cutoff, and the data is roughly uniformly distributed it is efficient to used a fixed cell size\cite{turk1989interactive}.

% Can be implemented anywhere there is a pair interaction term
Though the implicit solvent term was initially targeted, because it dominates the time spent in energy calculations, it is possible to apply this method to any pair-pair interaction.
Though especially for shorter range interactions it might be beneficial to either maintain a higher resolution hash, or implement a hierarchical spatial hash, such as an octree.
We are also applying this geometric hashing technique to collision detection between simultaneous loop predictions.
This will allow efficient screening of neighboring loop prediction candidates, which will be particularly useful in predicting structures with multiple nearby solvent exposed loops, such as G-protein coupled receptors.

% implicit/explicit comparison discussion
Implicit solvation models offer a very tangible benefit over explicit solvent models, both in performance and experimental complexity.
Though explicit solvation is sometimes viewed as a ``gold standard'', it has been shown that current implicit solvent models can, at least sometimes, reproduce predictions of explicit solvent models.
However, development of implicit solvent models is important since improved performance compared with explicit solvent methods allows modeling of larger systems, longer timescales, or improved sampling.
The complexity of the experiment is also reduced, using implicit solvent models, because results are not dependent on sampling of water conformations in addition to protein conformations.
However, implicit solvent models can still be very computationally expensive.
For instance, in the PLOP implementation of the OPLS-AA energy model with SG-B solvation term, evaluating the solvation term consumes up to 80\% of the total time spent in energy calculations, dependent on size of the symmetric system.
This is in large part due to the time complexity of evaluating the SG-B solvation term.
Thus an algorithm that offers further reduction in experimental cost without a tradeoff in accuracy represents valuable progress in the development of implicit solvent models.


%%%%%%%%%%%%%%%%%%%%%%%%%%%%%%%%%%%%%%%%%%%%%%%%%%%%%%%%%%%

% practical considerations and limitations
Because the size of protein systems are limited, at some level, by physical limits, the maximum system size expected to be encountered is limited.
Therefore, although the new method reduces the time complexity of implicit solvent calculations, the maximum expected speedup is limited to about a factor of three in large systems.
The actual speedup depends on both the system size, and the amount of time that a given experiment spends evaluating the solvent contribution.
The speedup observed in side chain prediction experiments was much less, around 20\%, though it is possible that applying a similar method to terms of the gradient during minimization would increase that amount.
Nonetheless, even a 20\%  speedup represents a significant improvement, especially as structure prediction methods continue to depend on parallel prediction and reprediction of the same region as a method of structure refinement\cite{goldfeld2013loop}.
% could be future work
Although this algorithm improves on the theoretical time complexity of the S-GB implicit solvent model, parameterization such as the number, and size of cells in the grid structure could have a significant effect on run time.
Some effort was made towards choosing reasonable parameters, but they are likely not optimal.
Hardware that is optimized for this sort of spatial indexing and collision detection, or proximity detection, exists in modern video cards, and along with general purpose programming for this sort of hardware it should be possible to further parallelize computation of implicit solvation effects for even greater performance improvements\cite{harris2008cuda}.

