% abstract should be roughly two pages
The process of bringing drugs to market continues to be a slow and expensive affair.
And despite recent advances in technology, the cost both in monetary terms and in terms of time between target identification and arrival of a new drug on the market continues to increase.

High throughput screening is a first step towards testing a large number of possible bioactive compounds very quickly.
However, the space of possible small molecules is limitless, and high throughput screening is limited both by the size of available libraries and the cost of running such a large number of experiments.
Therefore, advancements in computational drug screening are necessary in order to maintain the current rate of progress in modern medicine.

Computational drug design, or computer assisted drug design, offers a possible way of addressing some of the shortfalls of conventional high throughput screening.
Using computational methods, it is possible to estimate parameters such as binding affinity of any small molecule, even those not currently present in any small molecule library, without having to first invest in the often slow and expensive process of finding a synthetic pathway.
Computational methods can be used to screen similar molecules, or mutations in small molecule space, seeking to increase binding affinity to the protein target, and thereby efficacy, while simultaneously minimizing binding affinity to other proteins, decreasing cross reactivity, and reducing toxicity and harmful side effects.

Computational biology methods of drug research can be broadly classified in a number of different ways.
However, one of the most common classifications is according to the methods used to identify possible drug compounds and later optimize those leads.
The first broad category is informatics or artificial intelligence based approaches.
In these approaches, artificial intelligence methods such as neural networks, support vector machines, and qualitative structure-activity relationships (QSAR) are used to identify chemical or structural properties that contribute heavily to binding affinity.
The next category, ligand based approaches, is very useful when there are a large number of known binders for a specific family of proteins.
In this approach, the ligands are clustered using a metric of chemical similarity and new compounds which occupy a similar chemical space are likely to also bind strongly with the protein of interest.
The final class of methods of computational drug design, and the method explored in this thesis, is the diverse class known as structural methods.
These approaches in the most general sense make use of a sampling method to sample a number of protein, or protein-small-molecule interaction conformations and an energy model or scoring function to measure dimensions which would be very difficult and or expensive to measure experimentally.

In this thesis, a number of different sampling methods that are applicable to different questions in computational biology are presented.
Additionally, an improved algorithm for evaluating implicit solvent effects is presented, and a number of improvements in performance, reliability and utility of the molecular mechanics program used are discussed.
