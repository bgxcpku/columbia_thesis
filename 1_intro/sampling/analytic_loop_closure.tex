Subsequences with regular secondary structures, $\alpha$-helices and $\beta$-sheets are generally better conserved, and therefore likely to be well covered by simple homology models \cite{kolodny2005inverse}.
The intervening ``random coil'' or loop regions often play a large role in determining protein specificity for a specific ligand as in antigen-antibody binding \cite{bajorath1996comparison}, small protein toxins to the receptors they target \cite{wu1996functional}, or transcription factors to specific DNA sequences \cite{jones1999protein}.

Loop closure or prediction is a significant part of homology modeling, and building structures consistent with X-ray refraction data.
Therefore in order to accurately predict three dimensional structure through homology models, infer the protein binding partners and function, or even build a three dimensional structure consistent with both X-ray data and physical constraints, accurately predicting these loop regions is critical \cite{fiser2000modeling}.

The question is, given two fixed endpoints and a flexible loop, or actuator, find a conformation of the loop which connects the two endpoints.
Because of the similarities that this problem solves with robotics a number of algorithms have been adapted from that field \cite{kolodny2005inverse}.
The first of these is analytical loop closure, where a conformation which satisfies the closure criteria is solved for directly by solving a system of equations.
Though this problem can be solved analytically for small loops \cite{wedemeyer1999exact,go1970ring,bruccoleri1985chain,palmer1991standard}, the problem becomes more difficult as loop length grows and the number of degrees of freedom of the loop section increases.
Additionally these closure constraints make sampling multiple different conformations more difficult \cite{cortes2005sampling}, though it is possible to hierarchically solve sub-loops in order to generate conformations for possible loops \cite{wedemeyer1999exact}.
