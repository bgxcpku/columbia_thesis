Explicitly modeling each water molecule and sampling over possible conformations is the most realistic possible model.
However, this adds a huge number of degrees of freedom to the system, significantly complicating sampling, and also a large number of interactions to compute between water molecules.
Because of these complexities with the efficient methods of sampling explicit solvent models, these simulations are too expensive to use on systems the size of proteins \cite{figueirido1997large,zhang2001solvent}.

Therefore there is significant interest in continuum models which accurately describe the mean force of water, without requiring additional sampling or interactions as in explicit models \cite{zhang2001solvent,still1990semianalytical,qiu1997gb}.
These methods have the potential to be three orders of magnitude, or even more, faster than explicit solvent experiments \cite{zhang2001solvent}.
The total free energy of solvation can be separated into polar and non-polar components, which correspond to the work done inserting the uncharged solute molecule, or protein, into the solvent and then building the charges to their native values \cite{roux1999implicit}.


\cite{roux1999implicit}
