Hit compounds generally have a binding affinity for the target protein on the order of micromolar binding.
The goals of hit-to-lead optimization are to further increase that affinity with the goal of eventually reaching binding affinities on the order of \textapprox10 nanomolar or better, find other molecules with similar chemical characteristics to increase the size and diversity of the set of lead compounds, and screen hit compounds for any obvious issues.
At this stage of computational screening, more accurate energy models are required than for the initial screen \cite{jorgensen2004many,gohlke2002approaches,jorgensen2009efficient}.

In the hit-to-lead stage, there are multiple methods used to convert hit compounds into multiple and chemically distinct lead compounds.
First, pieces of multiple hit compounds can be joined to construct larger compounds, hopefully accumulating the attractive forces of each.
Second, functional groups can be added or replaced through molecular growing and evolution techniques.
Finally, a library can be searched by chemical similarity to the initial hit compound.
In all cases, the potential lead compound is docked or grown in the known binding site of the protein target.
Docking as a means of converting hit compounds to lead compounds is very similar to docking as a means of hit generation.
However, in this case the small molecule library is restricted to chemical space surrounding hit compounds.

A popular program for building, or mutating, lead compounds is Biochemical and Organic Model Builder (BOMB) \cite{barreiro2007docking}.
BOMB can operated as either a hit identification program or as a hit-to-lead optimization method.
Working to identify new compounds, BOMB starts with a number of different small ``core'' scaffolds and attempts to increase binding affinity by adding or replacing substituents to add favorable interactions while avoiding steric clashes.
BOMB has been successfully used to evolve a hit compound that showed no inhibition of HIV reverse transcriptase (RT) into a potent non-nucleoside RT inhibitor with nanomolar level binding \cite{barreiro2007docking}.

% which is hopefully well correlated with the binding energy,
% Interestingly, it is not necessarily the case that the scoring function is anchored in a physical force field.
% It is possible to use statistical or artificial intelligence approaches with success, so long as they are able to successfully solve the classification problem of distinguishing strong binders from weak binders.
After conversion of a hit compound into a number of possible leads, a scoring function is used to rank and identify lead compounds.
This scoring function may be based either on statistical knowledge of similar structures or basic physical forces.
A succssful scoring function, be it knowledge-based or physical, must be able to successfully solve the classification problem of distinguishing strong binders from weak binders.
For initial hit generation, a coarse grained energy function may be sufficient to differentiate ligands which bind strongly from those which do not bind at all.
However, in order to convert hit compounds to lead compounds, it is necessary to use a more sensitive (and generally slower) energy model to accurately rank the binding affinity of different small molecules \cite{jorgensen2004many,gohlke2002approaches}.
These energy models will be discussed in \nameref{section:energy_functions} (\ref{section:energy_functions}).

Whereas previously, lead compounds were evaluated almost exclusively on binding affinity to the target protein, recently more emphasis is being placed on identifying hit compounds that satisfy other characterisitcs besides binding affinity \cite{bleicher2003hit}.
It is important to begin to consider other characterisitcs of the potential drugs earlier in the pre-clinical process, because later it is difficult to make changes affecting characterisitcs such as solubility without significantly altering the binding affinity of an already highly modified hit compound.
As lead compounds are rarely very chemically distinct from the hits from which they were derived, and increasing binding affinity is actually sometimes an easier problem than addressing some of the other characteristics in the ``rule of five'', it is reasonable to begin by first trying to optimize hit compounds to satisfy some other criteria and postpone maximizing binding affinity \cite{proudfoot2002drugs}.
