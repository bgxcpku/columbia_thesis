The process of bringing new drugs to market is a long and expensive affair.
At the least, it is necessary to identify a possible target molecule, find a small molecule with promising binding characteristics to that target, and is additionally neither toxic nor a strong binder to the wide variety of other proteins necessary for regular cellular function.
These small molecules are then varied to maximize binding affinity to the target molecule, while attempting to simultaneously minimize cross reactivity.
Finally, after this process, these drug compounds are rigorously tested through clinical trials.

Information about both the costs and time necessary to bring a drug to clinical trials are less available than statistics for drug molecules reaching clinical trials.
As such, there is much debate over the average cost and time investment needed to develop a new drug.
The final costs necessary for the entire process range from \textapprox400 million US Dollars per new chemical entity to as much as \textapprox2 billion USD.
Estimates for the time required also vary significantly, but many estimates place the time required at around 10 years from target identification to an approved drug entering the market.
One of the largest factors affecting the average cost of each new drug compound is the low success rate in clinical trials for compounds that have been under active research for a number of years.
Because clinical trials are lengthy and expensive and and of themselves, but so too is the process leading to clinical trials, effectively screening these compounds earlier in the pipeline has the potential to significantly decrease the average cost of each new drug molecule \cite{adams2006estimating}.

The average cost of identifying a new drug molecule and gaining approval for that molecule is actually growing at a rate greater than inflation.
It is not just the rapidly growing costs of drug development that are alarming, but that these costs are increasingly ineffective and inefficient.
The number of new drugs introduced during the period 2005-2010 was actually 50\% fewer than the number introduced during the previous five years.
This decreased rate of drug discovery is disheartening because new drug compounds have been shown to have important impacts on both longevity and quality of life.
In fact, during the 14 year period from 1986 to 2000, 40\% of the two year increase in life expectancy can be accounted for by the effect of new drugs introduced during that period \cite{paul2010improve}.

The expected period of time that a candidate drug compound will spend in clinical trials is approximately nine to fourteen years \cite{dimasi2003price,paul2010improve}.
During the period from 1981 to 1990, the rate of approval of potential drugs decreased, as did that of self-originated drugs, or those drugs that were originally identified by a pharmaceutical research company.
Of potential drug compounds reaching clinical trials, only 10\% will finally be approved as new drugs \cite{dimasi2001risks,paul2010improve}.
Of potential drug compounds entering clinical trials that fail to be approved as new drugs, approximately 60\% will be abandoned or fail during phase II clinical trials \cite{dimasi2001risks}, which test the efficacy of a drug.
This is generally viewed as a failure to find a small molecule with sufficiently high binding affinity to the target protein.
Thirty percent of potential drug compounds entering clinical trials will fail in stage I, either because they are poorly tolerated, toxic to humans or cause side effects \cite{dimasi2001risks}.
Each of these is a potential indicator of cross reactivity with proteins other than the target molecule.

Finally, approximately 20\% of potential drug compounds entering clinical trials will fail in stage III \cite{dimasi2001risks}.
These drugs fail for a variety of reasons, though ineffectiveness is frequently cited as a reason.
All told, efficacy accounts for 37.6\% of all drugs that are abandoned after reaching clinical trials, making it the single largest contributing factor to the failure of these compounds to eventually receive approval as new drugs.
Other factors include safety, and economics \cite{dimasi2001risks}.

Computationally screening these compounds earlier in the process has the potential of reducing the attrition rate at this point in the process.
Additionally, increasing the affinity for the target itself can allow for lower dosages, which can increase survival of the drug candidate through phase II clinical studies.

For new chemical entities introduced in the 1990's, the cost of research and development is increasing at a rate 7.4\% above inflation.
Rates for the 2000's are not yet available or are only now becoming available due to the long lead time between introduction of a new chemical entity and that new chemical entity becoming an approved drug.
During the period from 1985 to 2000, the rate of spending on research and development increased at approximately twice the rate of introduction of new chemical entities.
Although the largest factors in determining this cost are the costs during clinical trials, significant amounts are also spent earlier in the drug discovery pipeline, such as target identification, lead identification, and lead optimization.
Improved computational techniques are generally viewed as possible means of decreasing costs or times associated with the earlier steps in the process.
However, by increasing the fraction of leads that survive the screening process, techniques that help identify and optimize lead molecules can have a very large effect on the cost of each new molecular entity.
Clinical trials consist of six sometimes overlapping stages, denoted 0 to V, though stages I to III are where the majority of drug molecules are abandoned.
Of the candidate compounds that enter clinical trials, only approximately 20\% will finally be approved as drugs \cite{dimasi2003price}.
Using computational methods to improve efficiency of pre-clinical drug development has the potential to not reduce the cost of developing new drugs, by ensuring that money spend in clinical trials is more likely to lead to marketable drugs.
Thus, the impact of computational drug design can reduce the costs of both the pre-clinical and clinical stages of drug development.  

Since 1950, the number of new chemical entities introduced per billion dollars has decreased by 50\% every 9 years.
Possible problems cited as contributing to this decrease in efficiency include:
\begin{enumerate}
 \item the ready availability of high quality and effective generic drugs as treatment options for many diseases,
 \item decreased risk tolerance among regulatory agencies,
 \item increased spending and personnel without understanding underlying relationships between spending and personnel and discovery of new compounds, and the long period of time between beginning research on a drug target and finally gaining approval for a new drug compound, and
 \item systematic overestimation of the efficacy of high throughput screening techniques relative to more classical techniques such as clinical science, and animal screening \cite{scannell2012diagnosing}.
\end{enumerate}

The high failure rates during clinical trials have been identified as one of the most critical factors in determining the overall costs of drug development \cite{bleicher2003hit}.

